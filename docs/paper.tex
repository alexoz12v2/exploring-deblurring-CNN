\documentclass[a4paper,10pt,twocolumn]{article}

% Page margins and layout
\usepackage[margin=2cm]{geometry}
\usepackage{multicol}
\setlength{\columnsep}{0.7cm}

% glossary
\usepackage[acronym, toc]{glossaries}
\makeglossary
\usepackage[backend=bibtex, style=numeric]{biblatex}
\addbibresource{references.bib}

% Font and encoding
% \usepackage[utf8]{inputenc} % only pdflatex
% \usepackage[T1]{fontenc} % only pdflatex
\usepackage{lmodern}

% Math
\usepackage{amsmath, amssymb, bm}

% Graphics and plots
\usepackage{graphicx}
\usepackage{caption}
\usepackage{subcaption}
\usepackage{float}
\usepackage{pgfplots}
\pgfplotsset{compat=1.18}

% Tables
\usepackage{booktabs}
\usepackage{multirow}
\usepackage{array}
\usepackage{tabularx}
\usepackage{caption}

% Lists
\usepackage{enumitem}
\setlist{nosep}

% Hyperlinks
\usepackage[hidelinks]{hyperref}

% Glossary entries
\newglossaryentry{shift-invariant}{
    name=shift-invariant,
    description={Un kernel \`e shift-invariant se e solo se la sua trasformata di Fourier non cambia con la traslazione del contenuto dell'immagine}
}
\newglossaryentry{blind-deconvolution}{
    name=blind-deconvolution,
    description={Metodo di estrazione di ground-truth image in cui il blur kernel \`e ignoto}
}
\newglossaryentry{non-blind-deconvolution}{
    name=non-blind-deconvolution,
    description={Metodo di estrazione di ground-truth image in cui il blur kernel \`e noto}
}

% Title and author
\title{\textbf{Rete Convoluzionale per Image Deblurring}}
\author{Gruppo}
% \date{\today}

\begin{document}

\twocolumn[
\maketitle
\begin{abstract}
    L'obiettivo del progetto \`e l'applicazione di una U-Net Convoluzionale al fine di migliorare la qualit\`a dell'immagine in input rimuovendo il Blur causato dal moto del soggetto acquisito
    (\textit{Motion Blur}) o causato dalla messa a fuoco dell'obiettivo (\textit{Focus Blur})
\end{abstract}
\vspace{1em}
]

\section{Introduction}\cite{convir}

\section{Theory and Traditional Approach}

Una immagine con Blur \`e modellata matematicamente come convoluzione tra ground-truth image latente e blur kernel, dove si quest'ultimo essere \textit{\gls{shift-invariant}}. In questo caso,
l'estrazione dell'immagine sharp \`e un problema di \textit{Image Deconvolution}, la quale \`e suddivisa in \textit{\Gls{non-blind-deconvolution}} e \textit{\Gls{blind-deconvolution}}.\par
Formulazione Matematica:

\begin{math}
    \bm{b} = \bm{i} * \bm{k} + \bm{n}
\end{math}

Dove:

\begin{itemize}[topsep=0pt, noitemsep]
    \item[] $\bm{b}$: Immagine con blur
    \item[] $\bm{i}$: Immagine \textit{ground-truth} latente
    \item[] $\bm{k}$: Blur Kernel
    \item[] $\bm{n}$: Rumore presente nell'immagine per contare imperfezioni causate dall'acquisizione (quantizzazione, saturazione del colore, risposta non linare della camera, ...) (Esempio: rumore gaussiano)
\end{itemize}

\paragraph*{Non-Blind Deconvolution}

In questa metodologia tradizionale, il blur kernel \`e noto a priori (Esempio: Point Spread Function Gaussiana per Blur senza direzione, Linea con direzione e lunghezza per Blur con direzione).\par
Uno dei primi metodi utilizzati in questa categoria, implementato come comparazione, \`e la \textit{Wiener Deconvolution}, il cui obiettivo \`e la ricerca di un filtro $\bm{g}$ tale che, tramite
convoluzione con l'immagine blurred $\bm{b}$. Espresso nel dominio di Fourier:

\begin{align}
    \hat{\bm{I}} &= \bm{G}\bm{B} \\
    \bm{G}       &= \frac{|\bm{K}|^2}{|\bm{K}|^2+\frac{1}{\mathrm{SNR}}} \frac{1}{\bm{K}}
\end{align}

Dove:

\begin{itemize}[topsep=0pt, noitemsep]
    \item[] $\bm{G}$ e $\bm{K}$: trasformate di Fourier di $\bm{g}$ e $\bm{k}$
    \item[] $\mathrm{SNR}$: Signal to noise ratio (infinitamente alto se rumore assente)
\end{itemize}

Un'implementazione di tale metodo di Deblurring si basa su un metodo di ottimizzazione convessa chiamato \textit{Alternating Direction Method of Multipliers} (ADMM)\footnote{\url{https://stanford.edu/class/ee367/reading/lecture6_notes.pdf}}

\paragraph*{Blind Deconvolution}

\section{Model Architecture}

\section{Observations}

\section{Results}

\printglossary[title=Glossary, toctitle=Glossary]

\printbibliography

\end{document}
